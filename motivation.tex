% ----------------------------------------------------------------------
\section{Motivation}
% ----------------------------------------------------------------------

% ----------------------------------------------------------------------
\begin{frame}{Motivation}
\begin{itemize}
  \vfill
  \item<1->\alert<1>{Goals}:
  \begin{itemize}
    \item Teach Answer Set Programming (ASP)
    \item Develop a methodology for ASP
  \end{itemize}
  \bigskip
  \bigskip
  \item<2->There is a \alert<2>{gap} between
  \begin{itemize}
    \item Stable model semantics for \alert<2>{logic} programs
    \item Answer Set \alert<2>{Programming}
  \end{itemize}
  \vfill
\end{itemize}
\end{frame}

% ----------------------------------------------------------------------
\begin{frame}{Both sides}
\begin{itemize}

  \item<1-> \alert<1>{Stable model semantics} for logic programs
  \begin{itemize}
    \item Very general syntax (e.g., negation, disjunction, aggregates)
    \item \alert<1>{Formal semantics} using reduct (or HT-logic)
    \item Translational semantics for some constructs (e.g., choices)
    \item Not constructive, pre-grounding
  \end{itemize}
  Great in theory: general, elegant, concise\ldots 
  \bigskip\bigskip
  \item<2-> \alert<2>{Answer Set Programming} \only<3>{\alert<3>{(often)}}
  \begin{itemize}
    \item Restricted syntax (limited recursion, no disjunction)
    \item \alert<2>{Informal semantics} using examples
    \item Direct interpretation of constructs (e.g., choices and constraints)
    \item Constructive (grounding as needed)
  \end{itemize}
  Great in practice: easy modeling language with effective solvers

\end{itemize}
\end{frame}

% ----------------------------------------------------------------------
\begin{frame}{The gap}
\begin{itemize}
  \item<1-> The explanations about Answer Set Programming are 
        only indirectly linked to the formal semantics:
        \alert<1>{where is the reduct?}
  \bigskip
  \item<2-> How do \alert<2>{the experts} bridge the gap?
  \begin{itemize}
    \item Programming methodology: restricted syntax, and rules in order
    \item Non-recursive negation: simplified using the Splitting Set Theorem
    \item Positive rules: iterate over $\T{P}$ 
    \item Choice rules and constraints: direct interpretation
  \end{itemize}
  \bigskip 
  \item<3-> \alert<3>{This talk:}
  Tutorial + formal semantics for Answer Set Programming
  \begin{itemize}
    \item Those features are made explicit in the semantics
    \item Easy and constructive (grounding as needed) 
    \item Graphical interpretation
  \end{itemize}
\end{itemize}
\end{frame}
