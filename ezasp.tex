% myinput
\newcommand{\myinput}[1]{
\ifx\inlibrary\undefined
  \input{#1}
\else
  \input{../#1}
\fi
}

% uncomment to add extra slides
\newcommand{\addextrainput}[0]{1}

\newcommand{\extrainput}[1]{
\ifx\addextrainput\undefined
  {}
\else
  \myinput{#1}
\fi
}

% ----------------------------------------------------------------------
\lecture{ezasp}{ezasp}
% ------------------------------
%\part{Ezasp}
% ----------------------------------------------------------------------
%\section{Ezasp}
% ------------------------------
\myinput{ezasp/macros}
\myinput{ezasp/summary}
\section{Motivation}
\extrainput{ezasp/tampere/asp}
\myinput{ezasp/motivation}
%\extrainput{ezasp/tampere/salesperson}
\begin{frame}{Introduction: Outline}
  \medskip
  \tableofcontents
\end{frame}
\section{Examples}
\myinput{ezasp/example1}
\myinput{ezasp/example2}
\againframe<9-13>{ezasp:summary}
\section{Variables}
\myinput{ezasp/example3}
\myinput{ezasp/example4}
\againframe<20-21>{ezasp:summary}
\section{Recursion}
\myinput{ezasp/example5}
\myinput{ezasp/example6}
\myinput{ezasp/traveling}
\myinput{ezasp/oddeven}
\againframe<30-31>{ezasp:summary}
\section{Recursion and Negation}
\myinput{ezasp/negative}
\againframe<40,41>{ezasp:summary}
\section{Summary}
\againframe<50>{ezasp:summary}
%\myinput{ezasp/whatisleft}
