% myinput
\newcommand{\myinput}[1]{
\ifx\inlibrary\undefined
  \input{#1}
\else
  \input{../#1}
\fi
}

% ----------------------------------------------------------------------
\lecture{ezasp}{ezasp}
% ------------------------------
%\part{Ezasp}
% ----------------------------------------------------------------------
%\section{Ezasp}
% ------------------------------
\myinput{ezasp/macros}
\myinput{ezasp/summary}
%\section{Motivation}
\myinput{ezasp/motivation}
\begin{frame}{Outline}
  \medskip
  \tableofcontents
\end{frame}
\section{Examples}
\myinput{ezasp/example1}
\myinput{ezasp/example2}
\againframe<10-13>{ezasp:summary}
\section{Variables}
\myinput{ezasp/example3}
\myinput{ezasp/example4}
\againframe<20-21>{ezasp:summary}
\section{Recursion}
\myinput{ezasp/example5}
\myinput{ezasp/example6}
\myinput{ezasp/traveling}
\myinput{ezasp/oddeven}
\againframe<30>{ezasp:summary}
\section{Recursion and Negation}
\myinput{ezasp/negative}
\againframe<40>{ezasp:summary}
\section{Summary}
\againframe<50>{ezasp:summary}
\myinput{ezasp/whatisleft}
%\input{ezasp/logic-programs-one}
%\input{ezasp/introduction}
%\input{ezasp/logic-programs-one}
%\input{ezasp/introduction}
